
	\documentclass[a4paper,titlepage]{article}
%packages
	\usepackage{amsmath, amssymb, amsfonts}
	\usepackage[ngerman]{babel}
	\usepackage[usenames]{color}
	\usepackage[utf8]{inputenc}
	\usepackage{listings}
	\usepackage{graphicx}
	%Kopf und Fusszeile
	\usepackage{fancyhdr}
	%Seitenabstaende (margins)
	\usepackage[left=3cm,right=2cm,top=2cm,bottom=2cm,includeheadfoot]{geometry}

%pagestyle fuer Kopf und Fusszeile
	\pagestyle{fancy}
	\fancyhf{}
	%Kopfzeile
		\fancyhead[L]{OpenGL EERT}
		\fancyhead[R]{Robert Schadek}
		\renewcommand{\headrulewidth}{0.5pt}
	%Fusszeile
		\fancyfoot[C]{Seite \thepage}
		\renewcommand{\footrulewidth}{0.5pt}
%Titel
\title{
	{\huge OpenGL EERT\\
	\huge - Einzelprojekt -\\
	{\large Universit\"at Oldenburg}\\
	\large Wintersemester 2008/2009\\
	%Abgabe: 18.01.2008\\
	}
	\date{\today}
	\author{\\
		\large 	Robert Schadek\\
	}
}

%Sonstiges
	% keine Einrckung bei neuen Abstzen
	\setlength{\parindent}{0em}



%Dukumentanfang
\begin{document}
%Titel erzeugen und danach eine neue Seite bgeinnen
\maketitle
\tableofcontents
\section{Einleitung, Projektziel}
Der Name meines Projektes lautet EERT. Dies steht f"ur EERT enhanced rendering technology.
Die Idee hinter dem Projekt ist jene, dass ich ein Programm schaffen wollte, welches neue
Szenen darstellen kann, ohne jedes mal neu kompiliert zu werden. Au"serdem wollte ich Frustum Culling
implementieren, da dies wie ich finde zu jeder Grafikanwendung geh"ort die Echtzeit f"ahig 
sein will. Wie die letztendliche Abgabe aussehen wird kann ich zu diesem Zeitpunkt noch nicht sagen, 
da sich die Szenen wie bereits angesporchen ohne "andern des Quellcodes anpassen lassen sollen. 
Aller Voraussicht werde ich aber eine Variante der beigelegten Testszene verwenden, da diese sich sehr gut
dazu eignet die F"ahigkeiten von EERT zu demonstrieren.
\section{Techniken}
\subsection{Allgemein}
Zu den verwendeten Techniken kann man allgemein sagen, dass ich nicht versucht habe, auf biegen und brechen, 
jede vorgestellte Technik aus der Vorlesung in mein Projekt einzubauen. Ich habe vielmehr versucht Techniken 
zu implementieren, die nicht vorgestellt wurden, aber doch zu jeder Grafik-Engine dazugeh"oren. Au"serdem 
wollte ich es erm"oglichen Szenen von gro"sen Polygonenumfang zu zeichnen. ($>$ 1Mil Dreiecke)

\subsection{Szenenbeschreibung}
Meine Szenenbeschreibung sieht so aus, dass ich mir eine Dateistruktur ausgedacht hab in der man speichern 
kann welches Object geladen wird, welche Texturen zu ihm geh"oren, wo sich die Objektinstanzen befinden, wie 
sich jene verschieben und rotieren usw. Und mir die Arbeit einfach zu machen habe ich mich mir das .obj Format 
als Vorbild genommen.

\subsection{Editor} 
Um nicht alle Informationen in der Szenendatei von Hand einzutragen, habe ich mit ein kleines Programm 
geschrieben welches mir eine Szenen zuf"allig anhand von bestimmen Parametern erzeugt.

\subsection{Objektloader}
Der Implementierte Objektloader ist in der Lage jede Form von Mesh zu laden, solange sie nur aus Dreiecken 
besteht. Wie oben bereits angedeutet lade ich Objekte des Typs .obj. Ich habe dieses Format gew"ahlt, da es 
recht einfach zu verstehen ist und f"ur meine Zwecke vollkommen ausreicht.

\subsection{Szenenmusik}
Damit die Szene nicht zu steril wirkt, habe ich mit Hilfe der Jlayer Libray die Wiedergabe von MP3 Datein
implementiert. 

\subsection{Per Pixel Lightning}
Wie die "Uberschrift es bereits andeutet, soll das fertige Programm die Beleutung durch Per-Pixel Lightning 
realisieren. Aus dem einfachen Grund, da dies zur Zeit stand der Technik ist. Sollte ich die Zeit finden, 
werde ich versuchen zus"atzlich, dass Per Pixel Lightning noch mit Normal-Maps zu verbinden.

\subsection{Objektinstanzen}
Unter Objektinstanzen hat man eine Technik zu verstehen, die es mir erm"oglichteinen Mesh nur einmal im 
Speicher zu halten, ihn aber an vielen Stellen der Szene mit verschiedenen Eigentschaften zu zeichnen. Dies 
ist Sinnvoll, die Objekte die ich in einer Szene laden alleine ca. 2MB gro"s sind. Nimmt man nun an ich w"urde 
diese Objekte nun 400 mal laden um sie an 400 Stellen zu zeichnen br"aucht ich alleine ca. 800MB Speicher nur 
f"ur die Objektdaten.\\
Dank der Objektinstanzen muss ich die Objektdaten nur einmal laden und f"ur jede Instanz einen Postions- sowie 
Rotationsvektor.

\subsection{UV-Texturing}
Beim UV-Texturing oder wie es auch genannt wird UV-Mapping, wird ein komplexes 3D Objekt derart auseinander 
gefallt, dass es sich auf einer 2D Fl"ache sprich Texture darstellen l"asst. Dies erm"oglicht es Objekte zu 
texturieren ohne dabei auf den Hilfsmittel von OpenGL zur"uckzugreifen. Dies macht Sinn, da es mit diesen 
Hilfsmitteln nicht m"oglich ist, die Texturen beliebig komplex auf Objekten abzubilden. Das UV-Mapping steht 
in enger Verbindung mit dem Objektloader, da beim erstellen der Objekte bereits die sogenante UV-Map erstellt 
werden muss. Dies kann dann sp"ater in einem Beliebigen Zeichnenprogramm bearbeitet werden.

\subsection{Level of Detail}
Level of Detail kann man als Mipmapping f"ur Dreiecke verstehen. Ich setze voraus, dass jedes Objekt welches 
in EERT darstellt werden soll, in sechs Aufl"osungen vorliegt. Beispielhaft von 10000 bis 300 Dreiecken. Dies 
mache ich mir so zu nutze, dass ich sage, wenn ein Objekt so weit von der Kamera entfernt ist, dass es nurnoch 
wenige Pixel auf dem Bildschirm einnimmt, brauch es nicht aus mehrere tausend Dreiecken bestehen, es reicht 
wenn es ein paar hundert sind.
Betrachtet man nun mehrere diese Stufen f"allt es dem Benutzer nicht auf, dass bei bestimmen Abst"anden von 
der Kamera eigentlich verschiedene Objekte gezeichnet werden.

\subsection{Shadow Volumes}
Als technisches Highlight soll in der finalen Version von EERT das Schattenverfahren Shadow Volumes 
implementiert werden. Im Prinzip ist dieses bereits in dieser Version implementiert, allerdings funktionert
es noch nicht. Darum ist es nicht aktiviert.

\subsection{Octree}
Die Idee hinter dem Octree ist es, den Raum in dem sich die zu zeichnen Objekte befinden in acht Unterr"aume 
aufzuteilen. Nachdem der Raum das erste mal aufgeteilt ist wird "uberpr"uft welche Objekte in welche Unterraum 
liegen. Danach wird jeder Unterraum wiederrum in acht Unter"aume aufgeteilt und es wird erneut "uberpr"uft 
welche Objekte sich in ihm befinden. Dies wird solange fortgef"uhrt bis eine bestimmte Rekursionstiefe 
erreicht ist. Es ist darauf zu achten was f"ur eine Rekursionstiefe man w"ahlt da die Anzahl der Knoten mit 
$8^n$ w"achst.\\\\

Beim Rendern "uberpr"uft man nun, ob sich die Root-node im Frustum der Camera befindet, ist dies nicht so ist 
der Rendervorgang bereits vorbei, da es ausgeschlossen ist das sich irgendeine weitere Node und somit 
irgendein Objekt im Frustum befindet. Sollte sich eine Node im Frustum befinden werde alle ihre Kinder 
"uberpr"uft. Dies geschieht solange bis die "Uberpr"uft Node keine Kinder mehr hat, sollte sie sich immernoch 
im Frustum befinden werden alle Objekte dies sich in ihr befinden gezeichnet.\\\\

Dies erm"oglicht ein "uberaus effektives Frustum Culling. Die Datenstruktur ist zudem so schnell aufzubauen, 
dass es m"öglich ist, diese jedem Frame neu aufzubauen. Somit k"onnen alle Objekte frei bewegt werden.

\section{Bedienung}
\subsection{Args Argumente}
F"ur die Ausf"uhrung von EERT ist es erforderlich zwei Argumente an die Laufzeitumgebung zu "ubergeben. 
Das erste Argument gibt den Dateinamen der Szenenfile an. Die einzige zur Zeit unterst"utzte Szene ist 
SuzannTest6.eob. Das zweite Argument gibt an ob EERT im Fullscreen-Mode starten soll. Damit dies geschieht 
muss das Argument Fullscreen hei"sen. Wird das zweite Argument nicht "ubergeben wird eine Standart-Aufl"osung 
ausgew"ahlt. Der empfohlende Aufruf von EERT sieht somit folgenderma"sen auf.\\\\ java -jar Eert SuzannTest6.eob

\subsection{EERT Steuerung}
Nach einem Click in das aktive Fenster l"asst sich die Kameraausrichtung durch klicken und gleichzeitiges 
Bewegen der Maus ver"andern. W A S D "andern die Position der Kamera. Zus"atzlich kann man sich mit der 
Leertaste sowie der C-Taste nach Oben und Unten bewegen. Die Taste U aktivert den Fullscreen-Mode, allerdings 
wird hierbei die Szene von vorne gestartet. Es lassen sich Laufzeitinformationen "uber die I-Taste einblenden.
\end{document}
